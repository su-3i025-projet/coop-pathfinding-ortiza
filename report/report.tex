\documentclass[12pt,a4paper]{article}

%########################### Preferences #################################


% ******** vmargin settings *********
\usepackage{vmargin} %This give you full control over the used page arae, it maybe not the idea od Latex to do so, but I wanted to reduce to amount of white space on the page
\setpapersize{A4}
\setmargins{3cm}%			%left edge
            {1.5cm}%        %top edge
           {14.7cm}%		%text width
           {23.42cm}%       %text hight
           {14pt}%			%header hight
           {1cm}%   	    %header distance
           {0pt}%			%footer hight
           {2cm}%    	    %footer distance

% ********* Font definiton ************
\usepackage{t1enc} % as usual
\usepackage[utf8]{inputenc} % as usual
\usepackage[french]{babel}
\usepackage{makeidx}
\usepackage[titletoc]{appendix}
\usepackage[shortlabels]{enumitem} 
\usepackage{listingsutf8}
\usepackage{csquotes}
\usepackage[table]{xcolor}
\usepackage{xspace}
\usepackage{hyperref}
\usepackage{xparse}
\usepackage{amssymb}
\usepackage{amsfonts}
\usepackage{mathtools, stmaryrd}
\usepackage{amsmath}
\usepackage[T1]{fontenc}
\usepackage[pdftex]{graphicx}

% ********* Graphics definition *******
\usepackage{eso-pic} % these two are required to add the little picture on top of every page
\usepackage{everyshi} % these two are required to add the little picture on top of every page
\usepackage[colorinlistoftodos,prependcaption,textsize=tiny]{todonotes}

\renewcommand{\appendixname}{Annexe}
\hypersetup{
    colorlinks  = true,
    linkcolor   = black,
    citecolor   = blue,
    urlcolor    = blue,
    linktocpage = false
}

\pagestyle{plain} % on headers or footers on the first page

\makeindex
\begin{document}

\begin{titlepage}
	\centering
	\includegraphics[width=0.30\textwidth]{logo.jpg}\par\vspace{1cm}
	{\scshape\LARGE Sorbonne Universit\'e \par}
	\vspace{1cm}
	{\scshape\Large 3I025 : Introduction \`a l'IA et la RO\par}
	\vspace{1.5cm}
	{\Large \bfseries Projet :\par}
	{\huge\bfseries Cooperative Pathfinding\par}
	\vspace{2cm}
	{\Large\itshape Angelo Ortiz\par}
	
	\vfill

% Bottom of the page
	{\large Licence d'Informatique\par}
	{\large Ann\'ee 2018/2019\par}
\end{titlepage}

%\newpage

\tableofcontents
\listoffigures
\listoftables

\newpage

\part*{Introduction}
\addcontentsline{toc}{part}{Introduction}

De nos jours, c'est un fait que les robots se sont r\'epandus dans les industries et ils ont ainsi remplac\'e, parfois compl\`etement, l'humain dans certaines t\^aches. 
Par exemple, ils sont de plus en plus utilis\'es dans les actes chirurgicaux, comme la laparoscopie, du fait de leur pr\'ecision. Pour d'autres t\^aches, comme le d\'eplacement de marchandises ou des objets lourds dans les usines, il est aussi plus int\'eressant de les attribuer aux robots.

Mettons-nous dans cette derni\`ere situation. Nous avons une entreprise ABC qui compte plusieurs robots charg\'es de d\'eplacer des marchandises \`a l'int\'erieur d'une usine. 
Il est \'evident que l'objectif est d'effectuer des d\'eplacements de mani\`ere efficace, i.e.\ des trajets plus courts. 
Supposons de plus que les objets \`a ramasser sont r\'epartis entre l'ensemble de robots de sorte qu'il n'y ait qu'un seul robot ciblant un objet.

La premi\`ere r\'eponse na\"ive au probl\`eme des d\'eplacements est de calculer, pour chaque entit\'e, le plus court chemin \`a son but et de les faire emprunter ces chemins. En effet, plusieurs algorithmes de tr\`es bonnes complexit\'es temporelles et spatiales  sont connus pour la recherche des plus courts chemins. Cependant, nous nous apercevons que cette approche est \'erron\'ee car elle a n\'eglig\'e les potentielles collisions qui peuvent survenir en cours de route.

Cela nous amene \`a concevoir une autre strat\'egie tenant compte des possibles collisions pour r\'esoudre ce probl\`eme. Il s'agit ainsi du probl\`eme de la recherche coop\'erative (en anglais \textit{Cooperative Pathfinding}).

Dans le cadre de l'UE 3I025 - Introduction \`a l'intelligence artificielle et la recherche op\'erationnelle, le sujet du premier projet propos\'e prend la forme du probl\`eme \textit{Cooperative Pathfinding} et consiste \`a le r\'esoudre au travers des diff\'erentes strat\'egies appel\'ees opportuniste, coop\'erative de base et coop\'erative avanc\'ee.

Dans la suite de ce document, je vais expliquer en d\'etail les sp\'ecifications du sujet, mes choix d'impl\'ementation et commenter mes r\'esultats obtenus.

\newpage
\part{Pr\'esetation}


L'environnement de travail de ce projet a \'et\'e \texttt{PySpriteWorld} qui
Ce projet a consist\'e \`a concevoir diff\'erentes strat\'egies pour 

\part{Strat\'egies}

\section{Opportunisme}
\textit{local repair}

\section{Coop\'eration na\"ive}
\textit{planification}

\section{Coop\'eration avanc\'ee}


\subsection*{Question 2}
s

	
\printindex
\end{document}



